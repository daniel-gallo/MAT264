\documentclass{beamer}
\usetheme{Warsaw}

\title{Error bounds for elliptic PDEs}
\author{Daniel Gallo}
\institute{University of Bergen}
\date{25 May 2021}

\usepackage{amsmath}
\usepackage{amssymb}
\usepackage{bm}
\def\R{\mathbb{R}}

\begin{document}
    \begin{frame}
        \titlepage
    \end{frame}

    \begin{frame}{The heat equation}
        The \textbf{temperature} depends on \textbf{position} and \textbf{time}
        \begin{align*}
            u \colon \R^{n + 1} &\to \R \\
            (x_1, \dots, x_n, t) &\mapsto u(x_1, \dots, x_n, t)
        \end{align*}
        The \textbf{temperature} satisfies the \textbf{heat equation}
        \begin{equation*}
            u_t = \Delta u
        \end{equation*}
    \end{frame}

    \begin{frame}{The steady state heat equation}
        We assume that $u_t = 0$
        \begin{itemize}
            \item For the \textbf{homogenius} case, we have \textbf{Laplace's equation}
            \begin{equation*}
                \Delta u = 0
            \end{equation*}
            \item For the \textbf{inhomogenius} case (there is a heat source), we use \textbf{Poisson's equation}
            \begin{equation*}
                \-k \Delta u = f \implies
                \begin{cases}
                    \nabla \cdot \bm{q} = f \\
                    \bm{q} = -k \nabla u
                \end{cases}
            \end{equation*}
            \begin{itemize}
                \item $f$ is the \textbf{heat-flux density} of the source
                \item $k$ is the \textbf{thermal conductivity}
                \item $\bm{q}$ is the \textbf{flux} TODO
            \end{itemize}
        \end{itemize}
    \end{frame}

    \begin{frame}{Elliptic PDEs}
        Second-order linear PDEs can be written as
        \begin{equation*}
            Au_{xx} + 2Bu_{xy} + Cu_{yy} + Du_x + Eu_y + Fu + G = 0
        \end{equation*}
        In our case,
        \begin{equation*}
            -ku_{xx} -ku_{yy} - f = 0
        \end{equation*}
        Since $B^2 - AC = -k^2 < 0$, we say that our PDE is \textbf{elliptic}
    \end{frame}

    \begin{frame}{The numerical method}
    \end{frame}
\end{document}